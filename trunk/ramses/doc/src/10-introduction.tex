\clearpage
\section{Introduction}

The RAMSES package is intended to be a versatile platform to develop
applications using Adaptive Mesh Refinement for computational
astrophysics. The current implementation allows solving the Euler
equations in presence of self-gravity and cooling, treated as additional
source terms in the momentum and energy equations. The RAMSES code can
be used on massively parallel architectures when properly linked to the
MPI library. It can also be used on single processor machines without
MPI. Output files are generated using native RAMSES Fortran unformatted
files. A suite of post-processing routines is delivered within the
present release, allowing the user to perform a simple analysis of the
generated output files.

\subsection{About This Guide}

The goal of this User's Guide is to provide a step-by-step tutorial in
using the RAMSES code. This guide will first describe a simple example of
its use. More complex set-up will be addressed at the end of the
document. Typical RAMSES users can be grouped into 3 categories:

\begin{itemize}
   \item Beginners: It is possible to execute RAMSES using only run
parameter files. The code is compiled once, and the user only modifies
the input file to perform various simulations. Cosmological simulations
can be performed quite efficiently in this way, using the initial
conditions generated by external packages such as \pkg{mpgrafic}.
   \item Intermediate users: For more complex applications, the user can
easily modify a small set of routines in order to specify specific
initial or boundary conditions. These routines are called ``patches''
and the code should be recompiled each time these routines are
modified.
   \item Advanced users: It is finally possible to modify the base
scheme, add new equations, or add new routines in order to modify the
default RAMSES application. This guide will not describe these advanced
features. In this case, a new documentation would be given separately.
\end{itemize}

\subsection{Getting RAMSES}

RAMSES software can be downloaded in the \emph{Codes} section from various web
sites, the most frequently updated ones being the Bitbucket repository 
\url{https://bitbucket.org/rteyssie/ramses} and
\url{http://www.ics.uzh.ch/~teyssier/ramses}.  It is freely distributed under the CeCILL
software license (see section \ref{sec:cecill} and
\url{http://www.cecill.info/}) according the French legal system \emph{for
non-commercial use only}. For commercial use of RAMSES, please contact the
author, but be prepared for a massive financial compensation !

\subsection{Main features}
RAMSES contains various algorithms designed for:

\begin{itemize}
   \item Cartesian AMR grids in 1D, 2D or 3D
   \item Solving the Poisson equation with a Multi-grid and a Conjugate
Gradient solver
   \item Using various Riemann solvers (Lax-Friedrich, HLLC, exact) for
adiabatic gas dynamics
   \item Computing collision-less particles (dark matter and stars)
dynamics using a PM code
   \item Computing the cooling and heating of a metal-rich plasma due to
atomic physics processes and an homogeneous UV background (Haardt and
Madau model).
   \item Implementing a model of star-formation based on a standard
Schmidt law with the traditional set of parameters.
   \item Implementing a model of supernovae-driven winds based on a
local Sedov blast wave solution.
\end{itemize}

All these features can be used and parameterized using the RAMSES
parameter file, based on the Fortran ``namelist'' format.

\subsection{Acknowledgements}

The development of the RAMSES code has been initiated and coordinated by
the author. The author would like to thank all collaborators who took an
active role in the development of this version. They are cited in
chronological order.

\begin{itemize}
   \item Matthias Gonzalez and Dominique Aubert (initial conditions)
   \item St\'ephane Colombi and St\'ephanie Courty (cooling and atomic
physics)
   \item Yann Rasera (star formation, post-processing)
   \item Yohan Dubois (supernovae feedback)
   \item Thomas Guillet (multigrid Poisson solver)
   \item S\'ebastien Fromang, Patrick Hennebelle and Emmanuel Dormy (MHD
solver).
   \item Philippe Wautelet and Philippe S\'eri\`es (code optimization)
\end{itemize}

I would like to thank my collaborators for helping me developing more
advanced versions of RAMSES, not yet available as complete releases,
since it is mostly work in progress.

\begin{itemize}
   \item Beno\^it Commer\c{c}on (thermal conduction)
   \item \'Edouard Audit and Dominique Aubert (radiative transfer)
   \item R\'emi Abgrall and Richard Saurel (multifluid)
\end{itemize}

\subsection{The CeCILL License}
\label{sec:cecill}

This software is under Copyright of CEA and its author, Romain Teyssier.

This software is governed by the CeCILL license under French law and
abiding by the rules of distribution of free software. You can use,
modify and/or redistribute the software under the terms of the CeCILL
license as circulated by CEA, CNRS and INRIA at the following URL:
\url{http://www.cecill.info/}.

As a counterpart to the access to the source code and rights to copy,
modify and redistribute granted by the license, users are provided only
with a limited warranty and the software's author, the holder of the
economic rights, and the successive licensors have only limited
liability.

In this respect, the user's attention is drawn to the risks associated
with loading, using, modifying and/or developing or reproducing the
software by the user in light of its specific status of free software,
that may mean that it is complicated to manipulate, and that also
therefore means that it is reserved for developers and experienced
professionals having in-depth IT knowledge. Users are therefore
encouraged to load and test the software's suitability as regards their
requirements in conditions enabling the security of their systems and/or
data to be ensured and, more generally, to use and operate it in the
same conditions as regards security.

The fact that you are presently reading this means that you have had
knowledge of the CeCILL license and that you accept its terms.

